\documentclass[12pt,a4paper]{scrartcl}
\usepackage[T2A]{fontenc}
\usepackage[utf8]{inputenc}
\usepackage[russian]{babel}
\usepackage{indentfirst}
\usepackage{misccorr}
\usepackage{graphicx}
\usepackage{amsmath}
\usepackage{listings}
\usepackage[left=2.5cm, right=1.5cm, top=2.5cm, bottom=2.5cm]{geometry}

\begin{document}
	\begin{titlepage}
		\begin{center}
			\large
			Государственное образовательное учреждение высшего профессионального образования\\
			“Московский государственный технический университет имени Н.Э.Баумана”
			\vspace{0.25cm}
			
			
			\textsc{Дисциплина: Анализ алгоритмов}\\[5mm]
			\vfill
			
			\textsc{Лабораторная работа № 5}\\[5mm]
			
			{\large Многопоточность}
			\bigskip
			
			
			Бутолин Александр Алексеевич\\
			Студент группы ИУ7-52Б
			\vfill		
			
		\end{center}
		\begin{center}
			2018 г.
		\end{center}
	\end{titlepage}
	
	
	\begin{center}
		\textbf{Введение}
		\label{sec:intro}
    \end{center}

    В связи с возрастающей потребностью решать задачи, связанные с обработкой матриц, такие как расчет новых координат тела в пространстве, растет необходимость в эффективных алгоритмах по работе с ними. 
	Перемножение матриц - одна из стандартных и наиболее используемых операций над матрицами, поэтому существует несколько алгоритмов, позволяющих произвести подобные вычисления (в данной работе рассматриваются классический алгоритм и алгоритм Винограда).
	Важную роль в решении подобных задач играет скорость, с которой производятся вычисления.
	Одним из способов улучшить данный показатель является многопоточность.
	В данной работе требуется на примере алгоритмов перемножения матриц изучить и применить на практике метод параллельных вычислений.
	
	Цель работы: изучить алгоритмы умножения матриц (классический и Винограда), а также провести сравнительный анализ с реализацией на потоках. 	
	Задачи: 
	
	\begin{enumerate}
		\item {реализовать ПО, включающее данные алгоритмы;} 
		\item {провести замеры времени выполнения алгоритмов;} 
		\item {описать и обосновать полученные результаты в отчете о выполненной лабораторной работе.}
    \end{enumerate}
    
    \newpage
	\section{Аналитический раздел}
	\label{sec:analitics}
	
	В этом разделе описаны алгоритмы, использованные в данной лабораторной работе.
	\subsection{Описание алгоритмов}
	\label{sec:analitics:alg}
	
	Умножение матриц — одна из основных операций над матрицами.
	Матрица, получаемая в результате операции умножения, называется их произведением \cite{Matrix_Mult}. 
	Рассмотрим стандартный алгоритм перемножения двух матриц.
	Пусть есть две матрицы A и B размера $a\cdot b$ и $c\cdot d$ соответственно. 
	Тогда, результатом из умножения будет матрица C размером $a\cdot d$, имеющая вид (\ref{eq:prodresult}):
	
	\begin{math}\label{eq:prodresult}
	C =\begin{bmatrix}
	c_11 & c_12 & ... & c_1d\\
	c_21 & c_22 & ... & c_2d\\
	...\\
	c_a1 & c_a2 & ... & c_ad
	\end{bmatrix}
	\end{math}
	
	Каждый элемент матрицы (\ref{eq:prodresult}) представляет собой скалярное произведение соответствующих строки и столбца исходных матриц. 
	Такое выражение можно просчитать заранее \cite{Coppersmith_Winograd}. 
	Рассмотрим два вектора: 
	
	\begin{equation}\label{eq:vector1}
	V = (v_1, v_2, v_3, v_4)
	\end{equation}
	
	и
	
	\begin{equation}\label{eq:vector2}
	W = (w_1, w_2, w_3, w_4)
	\end{equation}
	
	
	Их скалярное произведение: 
	
	\begin{equation}\label{eq:scalar1}
	V * W = v_1 \cdot w_1 + v_2 \cdot w_2 + v_3 \cdot w_3 + v_4 \cdot w_4.
	\end{equation}
	
	
	Это равенство можно переписать в виде: 
	
	\begin{equation}\label{eq:scalar2}
	V * W = (v_1 + w_2) \cdot (v_2 + w_1) + (v_3 + w_4) \cdot (v_4 + w_3) - v_1 \cdot v_2 - v_3 \cdot v_4 - w_1 \cdot w_2 - w_3 \cdot w_4
	\end{equation}
	
	Несмотря на то, что выражение (\ref{eq:scalar2}) требует больше вычисления, чем (\ref{eq:scalar1}), выражение в правой части последнего равенства (\ref{eq:scalar2}) допускает предварительную обработку\cite{Coppersmith_Winograd}. 
	Части этого выражения можно вычислить заранее и запомнить для каждой строки первой матрицы и для каждого столбца второй, что позволяет выполнять для каждого элемента лишь первые два умножения и последующие пять сложений, а также дополнительно два сложения.
	В этом и заключается алгоритм Винограда \cite{Coppersmith_Winograd_2}. 
	
	В настоящее время умножение матриц активно применяется при решения задач \cite{Matrix_Mult}:
	
	\begin{enumerate}
		\item {касающихся машинного обучения; }
		\item {преобразования координат тела на плоскости или в пространстве.} 	
	\end{enumerate}
	

	\subsection{Вывод}
	\label{sec:analitics:conclusion}
	
	Исследование предметной области показало, что существует несколько возможных алгоритмов
	перемножения матриц. Это дает возможность произвести их сравнение для определения их преимуществ и недостатков.
	Данная задача актуальна, в связи с тем, что операции над матрицами находят обширное применение.

    \newpage
	\section{Конструкторский раздел}
	\label{sec:construct}
	
	Для реализации алгоритмов были проведены следующие действия.

\subsection{Модель вычислений}
\label{sec:construct:model}

Введем модель вычислений, используемую при оценки трудоемкости:

\begin{enumerate}
	\item {объявление переменной, массива без определения имеет трудоемкость 0; }
	\item {операторы <знак>=, а также ++ и -- имеют трудоемкость 1; }
	\item {условный оператор имеет трудоемкость 0 \\ а само условие расчитывается по операциям; }
	\item {операции =, +, -, *, / имеют трудоемкость 1; }
	\item {операця индескации имеет трудоемкость 1; }
	\item {оператор цикла имеет трудоемкость 2 + n * (2 + тело цикла) \\ для цикла вида for (int i = 0; i < n; i++).}	
\end{enumerate}
	
	\subsection{Схемы алгоритмов}
	\label{sec:construct:schemes}
	
	На рисунках \ref{fig:first} и \ref{fig:second} представлены схемы реализуемых алгоритмов, а так же рассчеты трудоемкости каждого из них.
	
	\begin{center}
		\centering
		%\includegraphics[scale=0.6]{standart_alg.pdf}	
		%\captionof {figure}{Классический алгоритм перемножения матриц}
		\label{fig:first}
	\end{center}
	
	
	\begin{center}
		\centering
		%\includegraphics[scale=0.45]{Vinograd.pdf}
		%\captionof {figure}{Алгоритм Винограда перемножения матриц}
		\label{fig:second}
	\end{center}
	
	\newpage
	\subsection{Сравнительный анализ стандартного алгоритма перемножения матриц и алгоритма Винограда}
	\label{sec:construct:compare}
	
	Как видно из Рис. \ref{fig:second} алгоритм Винограда можно оптимизировать, так как некоторые операции можно вычислять заранее. 
 Если же сравнивать с классическим алгоритмом, можно сделать предположение, что алгоритм Винограда будет работать медленнее, чем стандартный, из-за наличия трех дополнительных циклов. 
 При этом, алгоритм Винограда использует дополнительные массивы, что должно увеличить объем потребляемой памяти. 
 
 По представленным схемам (раздел \ref{sec:construct:schemes}) можно произвести рассчеты трудоемкости. 
 Таким образом, трудоемкость классического алгоритма равна:
 
 \begin{equation}\label{eq:t1}
 T1 = 2 + n \cdot (2 + l \cdot (2 + 2 + m \cdot (2 + 8) )) \approx 10 \cdot n \cdot m \cdot l
 \end{equation}
 
 Трудоемкость алгоритма Винограда равна:
 
 \begin{equation}\label{eq:t2}
 \begin{gathered}
 T2 = 2 + n \cdot (2 + 2 + (m/2) \cdot (2 + 12)) + 2 + l \cdot (2 + 2 + (m/2) \cdot (2 + 12)) +\\
 2 + n \cdot (2 + 2 + l \cdot (2 + 7 + 2 + (m/2) \cdot (2 + 23))) + 3 + 2 + \\
 n \cdot (2 + 2 + l \cdot (2 + 13)) \approx 12.5 \cdot n \cdot m \cdot l
 \end{gathered}
 \end{equation}
 
 Трудоемкость оптимизированного алгоритма Винограда равна:
 
 \begin{equation}\label{eq:t3}
 \begin{gathered}
 T3 = 2 + (m/2) \cdot (2 + 2 + 2 + n \cdot (2 + 8) + 2 + l \cdot (2 + 8)) +\\
 2 + n \cdot (2 + 2 + l \cdot (2 + 7 + 2 + (m/2) \cdot (2 + 2 + 16))) + \\
 3 + 2 + 2 + n \cdot (2 + 2 + l \cdot (2 + 10)) \approx 10 \cdot n \cdot m \cdot l
 \end{gathered}
 \end{equation}
	
	
	\subsection{Вывод}
	\label{sec:construct:conclusion}
	
	Показанные в конструкторском разделе схемы позволяют реализовать представленные ими алгоритмы на любом языке программирования. 
	Также, проведенный сравнительный анализ двух алгоритмов умножения дал возможность предположить, что алгоритм Винограда будет работать медленнее, потребляя при этом больше памяти, чем классический алгоритм.
	Была рассчитана трудоемкость каждого алгоритма.
	
	\newpage
	\section{Технологический раздел}
	\label{sec:tech}
	
	В данном разделе приведена информация о конкретной реализации приведенных выше алгоритмов, а также исходный код полученных методов.
	
	\subsection{Требования к программному обеспечению}
	\label{sec:tech:demands}
	
	К программному обеспечению предъявлены следующие требования:
	\begin{enumerate}
		\item{возможность ввода размерностей матриц;}
		\item{возможность вывода результатов алгоритмов;}
		\item{возможность вывода замеров времени, затраченного на работу алгоритмов.}
	\end{enumerate}
	
	\subsection{Средства реализации}
    \label{sec:tech:relise}
    
    Лабораторная работа была выполнена в MonoDevelop на языке C \#. 
	Замеры процессорного времени были произведены с помощью встроенного метода TotalProcessorTime с последующим переводом в тики.
	
	\subsection{Листинг кода}
	\label{sec:tech:listing}
	
	Ниже приведены листинги реализованных методов.
	
	\lstset{ %
		language=C++,                 
		basicstyle=\small\sffamily, 
		numbers=left,               
		numberstyle=\tiny,           
		stepnumber=1,                   
		numbersep=5pt,            
		showspaces=false,            
		showstringspaces=false,      
		showtabs=false,            
		frame=single,              
		tabsize=2,                
		captionpos=t,               
		breaklines=true,           
		breakatwhitespace=false, 
		escapeinside={\%*}{*)}  
	}
	
	Листинг 1: Реализация классического алгоритма умножения матриц
	\lstinputlisting[language=C++]{Mult.cs}
	
	\newpage
	Листинг 2: Реализация алгоритма Винограда перемножения матриц
	\lstinputlisting[language=C++]{Mult.cs}
	
	\newpage
	Листинг 3: Реализация классического алгоритма умножения с использованием параллельных вычислений
	\lstinputlisting[language=C++]{Threading.cs}
	
	\newpage
	Листинг 4: Реализация алгоритма Винограда умножения с использованием параллельных вычислений
	\lstinputlisting[language=C++]{Threading.cs}
	
	\subsection{Вывод}
	\label{sec:tech:conclusion}
	
	Основываясь на схемах, полученных в конструкторском разделе, было реализовано три алгоритма перемножения матриц. 
	Программа была написана на языке C \# в среде MonoDevelop. 
	Были выполнены поставленные для ПО требования, а именно осуществлен ввод размерностей матриц и вывод результатов работы алгоритмов.

        \newpage
	
	\section{Экспериментальная часть}
	\label{sec:exp}
	
	При реализации алгоритмов была произведена проверка правильности работы. 	
	
	\subsection{Примеры работы}
	\label{sec:exp:examples}
	
	Были проведены тесты всех алгоритмов для определения правильности их работы.
	
	Тест 1. Умножение.
	
	Матрицы для перемножения:
	
	\begin{minipage}[c][3cm][c]{0,5\textwidth}
		\begin{math}\label{eq:test11}
		A =\begin{bmatrix}
		1 & 2 & 3\\
		-1 & 0 & 10\\
		0 & 0 & 0
		\end{bmatrix}
		\end{math}
	\end{minipage}
	\begin{minipage}[c][3cm][c]{0,5\textwidth}
		\begin{math}\label{eq:test12}
		B =\begin{bmatrix}
		2 & 4 & 0\\
		5 & 0 & -1\\
		0 & 1 & -1
		\end{bmatrix}
		\end{math}
	\end{minipage}
	
	Ожидаемый результат:
	
	\begin{center}
		\begin{math}\label{eq:res1}
		C =\begin{bmatrix}
		12 & 7 & -5\\
		-2 & 6 & -10\\
		0 & 0 & 0
		\end{bmatrix}
		\end{math}
	\end{center}
	
	Результаты, где (\ref{eq:res11}) - классический алгоритм и (\ref{eq:res12}) - алгоритм Винограда:
	
	\begin{center}
		\begin{minipage}[c][3cm][c]{0,3\textwidth}
			\begin{equation}\label{eq:res11}
			Classic =\begin{bmatrix}
			12 & 7 & -5\\
			-2 & 6 & -10\\
			0 & 0 & 0
			\end{bmatrix}
			\end{equation}
		\end{minipage}
		\begin{minipage}[c][3cm][c]{0,3\textwidth}
			\begin{equation}\label{eq:res12}
			Winogr =\begin{bmatrix}
			12 & 7 & -5\\
			-2 & 6 & -10\\
			0 & 0 & 0
			\end{bmatrix}
			\end{equation}
		\end{minipage}
	\end{center}
	
	Тест 2. Умножение на единичную матрицу.
	
	Матрицы для перемножения:
	
	\begin{minipage}[c][3cm][c]{0,5\textwidth}
		\begin{math}\label{eq:test21}
		A =\begin{bmatrix}
		1 & 0 \\
		0 & 1
		\end{bmatrix}
		\end{math}
	\end{minipage}
	\begin{minipage}[c][3cm][c]{0,5\textwidth}
		\begin{math}\label{eq:test22}
		B =\begin{bmatrix}
		2 & 4 \\
		5 & 0 
		\end{bmatrix}
		\end{math}
	\end{minipage}
	
	Ожидаемый результат:
	
	\begin{center}
		\begin{math}\label{eq:res2}
		C =\begin{bmatrix}
		2 & 4 \\
		5 & 0
		\end{bmatrix}
		\end{math}
	\end{center}
	
	Результаты, где (\ref{eq:res21}) - классический алгоритм и (\ref{eq:res22}) - алгоритм Винограда:
	
	\begin{center}
		\begin{minipage}[c][3cm][c]{0,3\textwidth}
			\begin{equation}\label{eq:res21}
			Classic =\begin{bmatrix}
			2 & 4 \\ 
			5 & 0
			\end{bmatrix}
			\end{equation}
		\end{minipage}	
		\begin{minipage}[c][3cm][c]{0,3\textwidth}
			\begin{equation}\label{eq:res22}
			Winogr =\begin{bmatrix}
			2 & 4 \\
			5 & 0
			\end{bmatrix}
			\end{equation}
		\end{minipage}
	\end{center}
	
	
	Тест 3. Умножение на нулевую матрицу.
	
	Матрицы для перемножения:
	
	\begin{minipage}[c][3cm][c]{0,5\textwidth}
		\begin{math}\label{eq:test31}
		A =\begin{bmatrix}
		-1 & 10 \\
		2 & 1
		\end{bmatrix}
		\end{math}
	\end{minipage}
	\begin{minipage}[c][3cm][c]{0,5\textwidth}
		\begin{math}\label{eq:test32}
		B =\begin{bmatrix}
		0 & 0 \\
		0 & 0 
		\end{bmatrix}
		\end{math}
	\end{minipage}
	
	Ожидаемый результат:
	
	\begin{center}
		\begin{math}\label{eq:res3}
		C =\begin{bmatrix}
		0 & 0 \\
		0 & 0 
		\end{bmatrix}
		\end{math}
	\end{center}
	
	Результаты, где (\ref{eq:res31}) - классический алгоритм и (\ref{eq:res32}) - алгоритм Винограда:
	
	\begin{center}
		\begin{minipage}[c][3cm][c]{0,3\textwidth}
			\begin{equation}\label{eq:res31}
			Classic =\begin{bmatrix}
			0 & 0 \\
			0 & 0 
			\end{bmatrix}
			\end{equation}
		\end{minipage}	
		\begin{minipage}[c][3cm][c]{0,3\textwidth}
			\begin{equation}\label{eq:res32}
			Winogr =\begin{bmatrix}
			0 & 0 \\
			0 & 0 
			\end{bmatrix}
			\end{equation}
		\end{minipage}
	\end{center}
	
	Тест 4. Умножение матриц 1x1.
	
	Матрицы для перемножения:
	
	\begin{minipage}[c][3cm][c]{0,5\textwidth}
		\begin{math}\label{eq:test41}
		A =\begin{bmatrix}
		-1
		\end{bmatrix}
		\end{math}
	\end{minipage}
	\begin{minipage}[c][3cm][c]{0,5\textwidth}
		\begin{math}\label{eq:test42}
		B =\begin{bmatrix}
		2 
		\end{bmatrix}
		\end{math}
	\end{minipage}
	
	Ожидаемый результат:
	
	\begin{center}
		\begin{math}\label{eq:res4}
		C =\begin{bmatrix}
		-2 
		\end{bmatrix}
		\end{math}
	\end{center}
	
	
	
	\begin{center}
		\begin{minipage}[c][3cm][c]{0,3\textwidth}
			\begin{equation}\label{eq:res41}
			Classic =\begin{bmatrix}
			-2 
			\end{bmatrix}
			\end{equation}
		\end{minipage}	
		\begin{minipage}[c][3cm][c]{0,3\textwidth}
			\begin{equation}\label{eq:res42}
			Winogr =\begin{bmatrix}
			-2 
			\end{bmatrix}
			\end{equation}
		\end{minipage}
    \end{center}
    
    Результаты, где (\ref{eq:res41}) - классический алгоритм и (\ref{eq:res42}) - алгоритм Винограда:
	
	
	\newpage
	\subsection{Постановка эксперимента}
	\label{sec:exp:setting}
Замеры времени производились на различной нечетной размерности матрицы от 11 до 1011 с шагом 100 для разного числа потоков (1, 2, 4, 8, 16).
Замеры процессорного времени были произведены с помощью встроенного метода TotalProcessorTime с последующим переводом в тики.

	
	\newpage
	\subsection{Сравнительный анализ на основе экспериментальных данных}
	\label{sec:exp:compare}
	
	По произведенным замерам были построены следующие графики.

\begin{center}
	\centering
	%\includegraphics[scale=0.75]{compare1.png}	
	%\captionof {figure}{Замеры для 1 потока}
	\label{fig:one}
\end{center}


\begin{center}
	\centering
	%\includegraphics[scale=0.75]{compare2.png}	
	%\captionof {figure}{Замеры для 2 потоков}
	\label{fig:two}
\end{center}


\begin{center}
	\centering
	%\includegraphics[scale=0.75]{compare4.png}	
	%\captionof {figure}{Замеры для 4 потоков}
	\label{fig:four}
\end{center}

\begin{center}
	\centering
	%\includegraphics[scale=0.75]{compare8.png}	
	%\captionof {figure}{Замеры для 8 потоков}
	\label{fig:eight}
\end{center}

\begin{center}
	\centering
	%\includegraphics[scale=0.75]{compare16.png}	
	%\captionof {figure}{Замеры для 16 потоков}
	\label{fig:sixteen}
\end{center}

Были также произведены замеры времени выполнения многопоточных реализаций, для определения зависимости времени выполнения от числа потоков. 
Выявленная зависимость представлена на рисунке %\ref{fig:ig:threads}.

\begin{center}
	\centering
	%\includegraphics[scale=0.8]{2.png}	
	%\captionof {figure}{Зависимость времени выполнения от числа потоков}
	\label{fig:threads}
\end{center}


\subsection{Вывод}
\label{sec:exp:conclusion}

Произведенные эксперименты показали, что параллельные вычисления в реализации классического алгоритма улучшают показатели, в то время как в реализации алгоритма Винограда ухудшают их, предположительно из-за времени создания объектов, копирования исходных данных и выполнения последнего цикла в основном потоке после завершения всех потоков.
Также был произведен анализ зависимости времени выполнения методов от числа потоков.

\newpage
\begin{center}
	\textbf{Заключение}
	\label{sec:outro}
\end{center}


В ходе лабораторной работы были реализованы два алгоритма перемножения матриц: классический и алгоритм Винограда.
Для обоих алгоритмов была предложена оптимизация, подразумевающая параллельные вычисления.
Был произведен анализ временных показателей для всех реализаций.
Результаты данного анализа, произведенного в экспериментальной части, показали, что параллельные вычисления улучшают работу классического алгоритма, в то время как распараллеленный алгоритм Винограда не улучшается в данной реализации, так как для каждого потока происходит копирование всех исходных данных, а последний цикл, предусмотренный алгоритмом для корректировки результата на нечетных размерностях матриц не модет быть распараллелен. Копирования элементов можно избежать, используя указатели, но это делает код менее безопасным и требует контроля памяти, так как автоматический сборщик мусора не контролирует указатели. 
Также было показано, что при совпадении числа потоков и числа логических ядер методы показывают самые лучшие результаты, после чего их показатели начинают ухудшаться.

    \newpage
    \begin{thebibliography}{9}
        \bibitem{Coppersmith_Winograd} Coppersmith and Shmuel Winograd. «Journal of Symbolic Computation» - М.:  Доклады Академий Наук СССР, 1965.

        \bibitem{Coppersmith_Winograd_2} «Алгоритм Копперсмита — Винограда» [Электронный ресурс]. Journal of Symbolic Computation. – Режим доступа:
        http://ru.math.wikia.com/wiki/Алгоритм Копперсмита — Винограда, свободный.	

        \bibitem{Matrix_Mult} Корн Г., Корн Т. Алгебра матриц и матричное исчисление // Справочник по математике. — 4-е издание. — М: Наука, 1978. 
    
        \bibitem METANIT.COM: Сайт о программировании. [Электронный ресурс]. \\URL:https://metanit.com/sharp/tutorial/11.1.php (дата обращения: 012.02.2019)
    \end{thebibliography}

\end{document}